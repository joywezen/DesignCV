%\newdateformat{myformat}{\THEDAY{ten }\monthname[\THEMONTH], \THEYEAR}

\begin{minipage}[t][2cm][t]{0.6\linewidth}
	%\vspace{1cm}
	\normalfont
Dr. Fahland \\[0.5em]
Bundesanstalt für Geowissenschaften und Rohstoffe (BGR)\\
Geozentrum Hannover\\
Stilleweg 2\\
30655 Hannover\\
%(bevorzugter Einsatzort: Braunschweig)
%Büro Braunschweig
	\vfill
\end{minipage}
\hfill
\begin{minipage}[t][5cm][t]{0.3\linewidth}
	\raggedleft
	\small{ \color{darkgray}
		Absender:\\\vspace{0.5em}Weiran Zhang\\ \vspace{0.5em}
		Vincent-van-Gogh-Ring 3D\\ 38126 Braunschweig\\\vspace{0.5em}
		weiran-zhang@outlook.com\\
		+49~(0)176 2136 5452\\}
	\vspace{0.5em}
Stellenausschreibung: B 112/20\\[0.5em]
 \today
	\vfill	
\end{minipage}
\begin{minipage}[t][1cm][t]{1\linewidth}
	\large\textbf{Bewerbung als wissenschaftlicher Mitarbeiter (B 112/20)\\[1em]}
	%% "Gekoppelte	Modellberechnungen Endlagerung"
\end{minipage}

\begin{adjustwidth}{0cm}{2cm}
	Sehr geehrte Frau Dr. Fahland,\\[1em]
	 am Anfang des Schreibens möchte ich mich bei Ihnen  für das freundliche Auswahlgespräch am 01.10.2020 bedanken. Damals hat Ihre Forschung für langzeitsichere Endlagerung radioaktiver Abfälle einen tiefen Eindruck bei mir hinterlassen. Besonders  das Mitteilungsschreiben von der Personalabteilung hat mich beeindruckt, dass die BGR für jede Bewerbung sehr verantwortlich ist. Auch bin ich mir davon bewusst, dass meine Kenntnisse über thermo-hydro-mechanisch gekoppelte Prozessen noch verbessert werden soll.\\[1em]
	  Vor einigen Tagen habe ich die mündliche Prüfung für meine Dissertation "Stochastische Modellierung und numerische Simulation von Ermüdungsschädigung" bei der Leibniz Universität Hannover (LUH) bestanden. Die Stellenausschreibung motiviert mich, eine erneute Bewerbung bei Ihnen einzureichen. Anbei eine Übersicht  über meine Person:\\[0em]	
	  	   %fundierte Kenntnisse  und Erfahrung mit numerischen Simulationen waren die Studienschwerpunkte meines Masterstudiums Computational Sciences in Engineering (CSE) an der TU Braunschweig. Seit 2016 arbeite ich als wissenschaftlicher Mitarbeiter an der Leibniz Universität Hannover (LUH) im Fachgebiet numerischer Mechanik mit Fokus auf Modellierung von thermodynamischen Prozessen. Ich finde die Stellausschreibung für meinen Karrierewechsel in die Industrie  äußerst interessant und passend. Anbei ein Überblick über meine Person:\\ 	   
	  %	   ich heiße Weiran Zhang und habe vor kurzem meine Dissertation mit Schwerpunkt auf FE-Berechnung von stochastischen Materialverhalten abgegeben. Meinen Master habe ich im Studiengang Computational Sciences in Engineering absolviert, in dem ich fundierte Kenntnisse und Erfahrung mit numerischen Simulationen sammeln konnte.  
		\begin{adjustwidth}{0.5cm}{0cm}
					\begin{itemize}
				\item[$\bullet$]	4 Jahre Erfahrung mit Finite-Elemente-Berechnung (FE) für Lebensdauerprognosen von Bauteilen am Institut für Baumechanik und numerische Mechanik an der LUH. Die Fachkenntnisse auf Themen wie Lösung von große lineare Gleichungssysteme, Beschleunigung der Berechnung 
				mit Computercluster sind vorhanden.  \\[-0.5em]
				%\item Erfahrung mit Simulation von thermo-mechanisch gekoppelten Prozessen während des Masterstudiengangs, z.B. das Arbitrary-Lagrangian-Eulerian-Verfahren (ALE) für Berechnung von Rollkontakten.\\
					\item[$\bullet$] Arbeitserfahrung mit Entwicklung von Schädigungsmodelle mittels stochastischen Prozessen im Labor LMT (Frankreich) mit besonderer Berücksichtigung auf 
					Anpassung von Modellberechnung an Messdaten. \\[-0.5em]
					
				\item[$\bullet$] Gutes Verständnis der Bayes-Statistik und deren Anwendungen wie parametrische Identifikation im Rahmen des Forschungsprojektes stochastischer Finite-Elemente am Institut für Wissenschaftliches Rechnen an der TU Braunschweig.\\[0em]		
		%	\item Erfahrung mit Oberflächenqualitätskontrolle der Triebwerkblätter durch CMM-Scanner aus Praktikum während der Bachelorphase. Gute Kenntnisse über  Total-Quality-Management (TQM) und deren statistischen Toolkits sind vorhanden. 
			%	\item[$\bullet$] Gute Kenntnisse in Total-Quality-Management (TQM) und deren statistischen Toolkits aus dem Bachelorstudium. Dabei sind Erfahrungen wie Qualitätskontrolle der Triebwerkblätteroberflächen durch CMM-Scanner, Servicequalitätsoptimierung mittels statistischer Auswertungsmethoden, etc. \\[-0.2em]	 
			%\item Erfahrung mit Multiphysikalischer Simulation während des Masterstudiengangs, z.B. Berechnung von THM-gekoppelten Prozessen. %und praktische  Erfahrung mit Tools der Statistical-Quality-Control (SQC) in Anwendung der Oberflächeninspektion von Turbinenschaufeln.
   			\end{itemize}
		\end{adjustwidth}
	In den Unterlagen sind aktueller Lebenslauf und ein Arbeitszeugnis zusätzlich von meiner vergangene Bewerbung zu erhalten. Da momentan noch einige Aufgaben für Veröffentlichung vom letzten Arbeitgeber zu fertigen sind, könnte ich frühesten am 01.01.2021 bei Ihnen anfangen.\\[1em]
	% Beispiele über die Erfahrung in den Bereichen Entwicklung und Simulation durch vergangene Projekte wurden anhand eines beigefügten Portfolios demonstriert.
	%Außerdem sind mein Lebenslauf, zwei Arbeitszeugnisse und Studienzeugnisse aus der Master- und Bachelorphasen in den Unterlagen zu erhalten. \\[1em]
	%Nach der Promotionsprüfung am 11.11.2020 könnte ich frühesten am 16.11.2020 bei Ihnen anfangen. Meine Gehaltsvorstellung liegt bei 56.000 Euro im Jahr. 
	Auf Ihre Rückmeldung freue mich sehr.\\[1.5em]
	Mit freundlichen Grüßen,\\[1em]
	\includegraphics[width=0.18\textwidth]{../../signature/zhang_signature.png}%\\[1em]
%	Weiran Zhang %(Berechnungsingenieur)
\end{adjustwidth}

