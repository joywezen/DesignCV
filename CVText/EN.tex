\newpage
\setlength{\parskip}{0.25em}
\renewcommand*{\thefootnote}{\fnsymbol{footnote}}
\setcounter{footnote}{0}
%\makecvtitle

\begin{textblock*}{21cm}(0cm,-0.08cm)	
	\includegraphics[scale=1]{./PageDesign/title.eps}
\end{textblock*}

\begin{textblock*}{21cm}(2cm,1.7cm)	
	\Huge \textcolor{white}{Weiran Zhang}
\end{textblock*}

\begin{textblock*}{21cm}(2cm,3.4cm)	
	\Large \textbf{Development, Simulation}
	%	\large{\textbf{Key Skills}}: Stochastic Modelling, Computational Mechanics \normalsize	
\end{textblock*}
\begin{textblock*}{21cm}(2cm,4.3cm)	
	%	Anschrift: Richterstr. 19 | 38106 Braunschweig | Deutschland\\
	\faXing~ \href{https://www.xing.com/profile/Weiran_Zhang4/cv}{Xing} ~ \faLinkedinSquare~ \href{https://www.linkedin.com/in/weiran-zh/}{Linkedin} ~ \faGlobe~\href{http://www.weiran.de}{weiran.de}\\
	\faEdit\hspace{0.45em}\href{mailto:weiran-zhang@outlook.com}{weiran-zhang@outlook.com} \\
	\faPhoneSquare~ +49~(0)176-2136 5452
	%Age: 25 | Civil status: Single | Nationality: Chinese
\end{textblock*}
\begin{textblock*}{21cm}(15cm,1.2cm)	
	%\includegraphics[scale=0.055]{picture2}
	%\noindent\shadowimage[scale=0.055]{picture2}\par\bigskip
	\begin{tikzpicture}[overlay]
	\node[opacity=1] at (2.2,-2.7)%%2.5,-2.3
	{\includegraphics[scale=0.0574]{../../signature/potrait-s_Zhang.jpg}};
	\end{tikzpicture}
\end{textblock*}
\begin{textblock*}{21cm}(16.3cm,25cm)	
	%\includegraphics[scale=0.055]{picture2}
	%\noindent\shadowimage[scale=0.055]{picture2}\par\bigskip
	\begin{tikzpicture}
	%\node[opacity=0.1] {\includegraphics[scale=1]{Bosch.eps}};
	\end{tikzpicture}
\end{textblock*}
\vspace*{9em} 	
	\section{Working Experiences}
\cventry{\normalfont (Nov 2016- Oct 2019)}{Research Assistant}{}{Institute of Mechanics and Computational Mechanics}{Leibniz University Hannover, Germany}{
	\textit{\textbf{*Topics:}}  \textbf{Stochastic reliability analysis}, \textbf{Computational mechanics}
	\begin{itemize}%
		%\item[\textit{Tasks}:] %\begin{itemize} DFG founded international training project
			\item Monte-Carlo simulation of stochastic process for structure reliability assessment.
			\item High performance  non-linear FE analysis for high-cycle fatigue (one million cycles in minutes).
			%\item Large scale data visualisation.
%			\item  with life-cycle prediction.
%		\end{itemize} 
 	\end{itemize}}
 \cventry{\normalfont (2017- 2019 for 6 months) }{Research Exchange in EU}{}{LMT Cachan, \newline Universit\'e Paris-Saclay}{France}{
\textbf{\textit{Topics:} Data analysis, Optical measurement, Algorithm optimisation}  
\begin{itemize}
%	\item Infrared and visible-light measurement of temperature distribution and crack propagation.
	\item Optical measurement of material deformation and crack propagation.
	\item Visualisation, statistics and regression analysis of fatigue scatters.
\end{itemize}
}
\cventry{\normalfont (Mar 2015- Jul 2015)}{Research Assistant}{}{Institute of Scientific Computing}{\newline Technical University Braunschweig, Germany}{
\textbf{\textit{Topics:} Multi-sensor data fusion, Bayesian statistics}  	\begin{itemize}%
		\item Parameter identification with ensemble Kalman filter (EnKF).
		\item Data assimilation with 4D-Var and EnKF methods.
		%Reference solution with open-source FE software FEAPpv (from UC Berkeley).%Geometrische Parameter-Identifikation aus einer Stahlplatte unter Spannung und Kompressionstest. Das Ziel war um die Position und Größe einer kreisförmigen Aufnahme in einer Stahlplatte zu identifizieren. Der Stand und Messdaten wurden auf der Open-Source Finite-Elemente Software FEAP (UC Berkeley) simuliert, doch der Posterior wurde durch EnKF identifiziert.  %For the simulation is used the ensemble Kalman filter and FE software FEAP (UC Berkeley).
\end{itemize}}
\cventry{\normalfont (Jun 2013- Sept 2013)}{Process Engineer}{}{Yutong Bus}{Zhengzhou, China}{\begin{itemize}%
		\item Statistical quality control of electrophoresis treating process.
\end{itemize}}
\cventry{\normalfont (Jun 2012- Jul 2012)}{Surface Quality Engineer}{intern}{Xi'an Aero-Engine}{Xi'an, China}{\begin{itemize}%
		\item  Optical inspection of aero-engine blades with CMM.
\end{itemize}}


\cventry{\normalfont (Jul 2009- Jun 2013)}{IT Administrator}{part-time}{Yuke Technology}{Zhengyang, China}{
	\begin{itemize}%
			\item Linux server maintenance,  LAN construction and client control. 
\end{itemize}}
\cventry{\normalfont (Jan 2011- Jun 2013)}{Hardware Technician}{part-time}{Hasee Computer}{Zhengzhou, China}{
	\begin{itemize}%
		\item Hardware maintenance and data recovery, construction of safety system.
	\end{itemize}
}
\footnotetext{* Demonstrated by the attached portfolio.}
\section{University Projects}
%\hspace*{-0.4cm}
\cventry{\normalfont (Oct 2015- Apr 2016)}{*Master Thesis}{German Note: \textbf{1,3}}{}{TU Braunschweig}{
	\textit{\textbf{Topic}}: \textbf{Uncertainty Quantification} of damage in RC beam using multilevel Monte-Carlo method
	\begin{itemize}%
		\item Elasto-damage reinforced concrete model with perfect bounding hypothesis,
		\item Modelling of uncertain material properties using random field and spectral approximation,
		\item Application and comparison of multilevel and classical Monte-Carlo methods in efficiency and accuracy.
\end{itemize}} 
%\footnotetext{* see attached portfolio for demonstration.}
\cventry{\normalfont (Nov 2014- May 2015)}{*Project Thesis} {German Note: \textbf{1,0}}{}{TU Braunschweig}{
	\textit{\textbf{Topic}}: \textbf{Data assimilation} with ensemble Kalman filter and four-dimensional variation method
	\begin{itemize}%
		\item Comparison on the efficiency and accuracy of the 4D-Var and EnKF methods.
		\item 4th order Runge-Kutta solution of multi-dimensional chaotic system.
\end{itemize}}
\cventry{\normalfont (Dec 2012- May 2013)}{Bachelor Project}{record: \textbf{excellent}}{Zhengzhou University of Aeronautics}{}{\textit{\textbf{Topic}}: \textbf{Quality optimisation} of banking service in less-developed region
	\begin{itemize}%
		\item Principal component analysis of service quality rating model.
		\item Topologie based service network optimisation.
		\item  Ergonomie improvement of ATM accessibility.
\end{itemize}}

\newpage
\footnotetext{* Two reference letters from former employers attached.} 
\section{Academic Education}
\cventry{\normalfont (Nov 2016- Sept 2020)}{\textbf{Promotion in Computational Mechanics}}{\textit{Defence date: 11.11.2020}}{\newline Leibniz University Hannover, Germany}{}{\textbf{\textit{Dissertation:}} Stochastic Modelling and Numerical Simulation of fatigue damage.}
\cventry{\normalfont (Oct 2013- May 2016)}{*M.Sc.}{}{\textbf{Computational Sciences in Engineering (CSE)}}{\newline Technical University Braunschweig, Germany}{
	\begin{itemize}
		\item Applied mathematics, mechanics and statistics.
				\item  Modelling and scientific computing.
		\item Master thesis and project thesis in excellent record.
\end{itemize}} 
%*~M.Sc.%\footnotetext{* two reference letters from research institute and company are attached below.}
\cventry{\normalfont (Sept 2009- Jun 2013)}{B.Eng.}{}{\textbf{Quality and Reliability Engineering}}{\newline Zhengzhou University of Aeronautics, China}{\begin{itemize}
		\item ISO 9000/14000 standards.
		\item Total quality management (TQM).
		\item Toolkit for reliability assessment, statistical process control. 
\end{itemize}}
\section{Further Skills and Interests}
\cvitem{Coding}{Matlab, C++ (Python test, QT), Abaqus (manipulated by Python script)}{}{}
%							\cvitem{\hspace{-1cm}Programming}{Matlab, Python}{}{}
\cvitem{Language}{English (fluent), German (fluent), Mandarin (native)}{}{}
%\cvitem{Language}{German (working language), Chinese (native), French \& Portugiese (entry level) }{}{}
%	\cvitem{}{\hspace{5.4em}EU and Chinese driving license available (without any fines/penalty in the past 5 years)}{}{}
%\cvitem{Licenses}{EU and chinese driving license}{}{}
\cvitem{Hobbies}{Sport fishing, Saxophone.}{}{}
%\hspace{-2em}*\hspace{2em} 
%, EU and Chinese driving license
%\begin{textblock*}{2cm}(2cm,18.1cm)
%*  
%\end{textblock*}

\section{Presentation at International Academic Events}

\cventry{\normalfont (May 2018)}{Poitier, France}{}{12th International Fatigue Congress}{}{\textit{Presentation: On the time discritisation effect of stochastic damage evolution.}}
\cventry{\normalfont (Jun 2019)}{Crete, Greece}{}{3rd International Conference on Uncertainty Quantification in Computational Sciences and Engineering}{}{\textit{Presentation: Stochastic modelling of fatigue process.}}
\cventry{\normalfont (Sept 2019)}{Barcelona, Spain}{}{15th International Conference on Computational Plasticity}{}{\textit{Presentation: Modelling approach and efficient numerical scheme for stochastic fatigue process.}}
\section{Publication}
Stochastic Material Modelling for Fatigue Damage Analysis. In: Virtual Design and Validation. Springer, Cham, 2020. S. 329-347. 
  % arguments 3 to 6 can be left empty
														\section{Awards}
								\cventry{\normalfont (Nov 2013)}{CSE "Junior"-Scholarship}{}{}{}{Technical University Braunschweig. Braunschweig, Deutschland}
								\cventry{\normalfont (Jun 2013)}{Excellent Graduate}{}{}{}{Zhengzhou University of Aeronautics. Zhengzhou, China}
								\cventry{\normalfont (Oct 2012)}{University Scholarship}{}{}{}{Zhengzhou University of Aeronautics. Zhengzhou, China}
								\cventry{\normalfont (Sept 2011)}{First Place of Undergraduates Career Plan Competition}{}{}{}{Zhengzhou University of Aeronautics. Zhengzhou, China}
								\cventry{\normalfont (Jun 2011)}{Second Place of University Music Competition of Henan Province}{}{}{}{Musician Association of Henan Province, Zhengzhou, China}
								\clearpage