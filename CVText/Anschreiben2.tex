%\newdateformat{myformat}{\THEDAY{ten }\monthname[\THEMONTH], \THEYEAR}
%\renewcommand{\familydefault}{\sfdefault}
\label{sec:Anschreiben}
\begin{minipage}[t][2cm][t]{0.6\linewidth}
	%\vspace{1cm}
	\normalfont
Dr. Fahland \\[0.5em]
Bundesanstalt für Geowissenschaften und Rohstoffe (BGR)\\
Geozentrum Hannover\\
Stilleweg 2\\
30655 Hannover\\
%(bevorzugter Einsatzort: Braunschweig)
%Büro Braunschweig
	\vfill
\end{minipage}
\hfill
\begin{minipage}[t][4.5cm][t]{0.3\linewidth}
	\raggedleft
	\small{ \color{darkgray}
		Absender:\\\vspace{0.5em}Weiran Zhang\\ \vspace{0.5em}
		Vincent-van-Gogh-Ring 3D\\ 38126 Braunschweig\\\vspace{0.5em}
		weiran-zhang@outlook.com\\
		+49~(0)176 2136 5452\\}
	\vspace{0.5em}
Stellenausschreibung: B 112/20\\[0.5em]
 \today
	\vfill	
\end{minipage}
\begin{minipage}[t][1cm][t]{1\linewidth}
	\large\textbf{Bewerbung als wissenschaftlicher Mitarbeiter (B 112/20)\\[1em]}
	%% "Gekoppelte	Modellberechnungen Endlagerung"
\end{minipage}

\begin{adjustwidth}{0cm}{1cm}
	Sehr geehrte Frau Dr. Fahland,\\[1em]
	 zuerst möchte ich mich bei Ihnen  für das freundliche Auswahlgespräch am 01.10.2020 bedanken. Ich habe weiterhin ein großes Interesse an der Forschung für langzeitsichere Endlagerung radioaktiver Abfälle, da dies ein entscheidendes Thema für eine sichere Zukunft ist. %Besonders hat das Mitteilungsschreiben von der Personalabteilung  mich beeindruckt, dass die BGR für jede Bewerbung sehr verantwortlich ist. Auch bin ich mir davon bewusst, dass meine Kenntnisse über thermo-hydro-mechanisch (THM) gekoppelte Prozessen noch verbessert werden soll.
	 Die Vielfältigkeit des Themas und dessen Anwendungsbereich motivieren mich meine Kenntnisse über thermo-hydro-mechanisch (THM) gekoppelte Prozessen zu verbessern und erweitern.\\[1em]
	  Vor einigen Tagen habe ich die mündliche Prüfung für meine Dissertation "Stochastische Modellierung und numerische Simulation von Ermüdungsschädigung" an der Leibniz Universität Hannover (LUH) abgeschlossen. Momentan bin ich auch auf der Suche nach neuen Forschungsstellen und bin dadurch auf Ihre Stellenausschreibung gestoßen, welche sich sehr gut mit meinem Interessengebiet im Bereich der Forschung deckt. %  Forschung zu treiben und anwendungsorientiert zu arbeiten, bin ich auf diese Stellenausschreibung gestoßen. %Die Stellenausschreibung motiviert mich, eine erneute Bewerbung bei Ihnen einzureichen. 
	 % Mein Plan für die Postdoc-Phase ist mit der Fachkollegen jedes Jahr neue Veröffentlichungen zu haben, damit meine Berufsentwicklung in der Akademie nachvollziehbar ist. Die Aufgaben der Stelle haben einigen Zusammenhang mit meinen bisherigen Erfahrungen, als Beispiele:\\[-0.5em]
	 Ich konnte einige Zusammenhänge zwischen den beschriebenen Aufgaben und meinen bisherigen Erfahrungen finden und möchte Ihnen diese  erläutern:\\[-0.5em]
		\begin{adjustwidth}{0.5cm}{0cm}
					\begin{itemize}	
	\item[$\bullet$] Das Kluftnetz und die Schichtungen des porösen Mediums könnte durch einen Zufallsfeld erstellt werden. Ein geeignetes numerisches Verfahren wäre z.B. die Karhunen–Loève Transformation. Dieses habe ich in meiner Masterarbeit implementiert, um zufällige Mikrostruktur im Beton nachzubilden.\\[-0.5em]
	\item[$\bullet$] Meine Untersuchungen während der Doktorarbeit über Finite-Elemente-Methode (FEM) von der lokalen, zufälligen Schädigungsentwicklung als Ergebnis mechanischen Ermüdungsprozessen zeigt, dass die Lebensdauerprognose aus Modellberechnung in der makroskopischen Skala mit der Laborprüfung vergleichbar ist. Die Integration der stochastischen Prozessen in der Phasenfeldmethode könnte ein innovativer Ansatz sein, um die zufällige Rissentwicklung in den porösen Medien unter THM-Belastung zu modellieren.\\[-0.5em]   
	\item[$\bullet$] Mehrskalenproblem behandelte ich bei der Modellierung der mesoskopischen Strukturen des Betons, in dem die Kontakt zwischen  Bindemittel und Zuschlagstoff mit zufälliger Geometrie berücksichtigt wurde. Ein statistisches Homogenisierungsverfahren wird benutzt um die Beanspruchungs-zustände der Representative-Elementary-Volume (REV) zu quantifizieren. Die Übertragung der Randbedingung aus verschiedenen Skalen in der FEM-Implementierung ist  einer der  anspruchsvollen Lösungsansätze. \\[-0.5em] 
	%der porösen Media im Geosystem ist notwendig   die Skale-Transition, statitiscal homogenisation
   			\end{itemize}
		\end{adjustwidth}
	Einen weiteren Einblick über meinen Werdegang erhalten Sie durch mein aktuelles Arbeitszeugnis, meinen Lebenslauf und weitere Zertifikate. Falls Sie Fragen dazu haben, stehe ich Ihnen gerne zur Verfügung. Für den Betritt der Stelle bin ich am 01.01.2021 verfügbar.	Auf Ihre Rückmeldung würde ich mich sehr freuen.\\[1.5em] 
	%In den Unterlagen sind aktueller Lebenslauf und ein Arbeitszeugnis zusätzlich von meiner vergangene Bewerbung zu erhalten. Da momentan noch einige Aufgaben für Veröffentlichung vom letzten Arbeitgeber zu fertigen sind, könnte ich frühesten am 01.01.2021 bei Ihnen anfangen.\\[1em]
	% Beispiele über die Erfahrung in den Bereichen Entwicklung und Simulation durch vergangene Projekte wurden anhand eines beigefügten Portfolios demonstriert.
	%Außerdem sind mein Lebenslauf, zwei Arbeitszeugnisse und Studienzeugnisse aus der Master- und Bachelorphasen in den Unterlagen zu erhalten. \\[1em]
	%Nach der Promotionsprüfung am 11.11.2020 könnte ich frühesten am 16.11.2020 bei Ihnen anfangen. Meine Gehaltsvorstellung liegt bei 56.000 Euro im Jahr. 
	Mit freundlichen Grüßen,\\[1em]
	\includegraphics[width=0.18\textwidth]{../../signature/zhang_signature.png}%\\[1em]
%	Weiran Zhang %(Berechnungsingenieur)
\end{adjustwidth}

