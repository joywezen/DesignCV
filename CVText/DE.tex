\newpage
\setlength{\parskip}{0.25em}
\renewcommand*{\thefootnote}{\fnsymbol{footnote}}
\setcounter{footnote}{0}
%\makecvtitle
\newcommand*{\cventrynocomma}[7][.25em]{%
	\cvitem[#1]{#2}{%
		{\bfseries#3}%
		\ifthenelse{\equal{#4}{}}{}{ {#4}}%
		\ifthenelse{\equal{#5}{}}{}{ \newline\small #5 \normalfont}%
		\ifthenelse{\equal{#6}{}}{}{ #6}%
		\strut%
		\ifx&#7&%
		\else{\newline{}\begin{minipage}[t]{\linewidth}\small#7\end{minipage}}\fi}}



% ###Design A
%\begin{textblock*}{21cm}(0cm,0cm)	
%	\includegraphics[scale=1]{./PageDesign/title-eps-converted-d.pdf}
%\end{textblock*}
%\begin{textblock*}{21cm}(2cm,3.4cm)	
%	\Large \textbf{Entwicklung, Simulation}\\[0.8em]
%%	\large{\textbf{Key Skills}}: Stochastic Modelling, Computational Mechanics \normalsize	
%\normalsize
%	\faXing~ \href{https://www.xing.com/profile/Weiran_Zhang4/cv}{Xing} ~ \faLinkedinSquare~ \href{https://www.linkedin.com/in/weiran-zh/}{Linkedin} ~
%	\faCodeFork~ \href{https://github.com/joywezen?tab=repositories}{GitHub}\\
%	% \faGlobe~\href{http://www.weiran.de}{weiran.de}\\		 
%	\faEdit\hspace{0.45em}\href{mailto:weiran-zhang@outlook.com}{weiran-zhang@outlook.com} \\
%	 \faPhoneSquare~ +49~(0)xxxxxx\\	 
%	 %\faMapMarker~~ Braunschweig~~ 
%	 \faWordpress~ \href{http://www.weiran.de}{weiran.de}
%\end{textblock*}

%###Design B



\begin{textblock*}{21cm}(-5cm,0cm)	
	\section{CV}
	%	\includegraphics[scale=1]{./PageDesign/title-eps-converted-a.pdf}
\end{textblock*}

\begin{textblock*}{21cm}(0cm,0cm)	
	%	\includegraphics[scale=1]{./PageDesign/title-eps-converted-a.pdf}
\end{textblock*}
%\definecolor{ultramarine}{RGB}{0,32,96}
\begin{textblock*}{21cm}(2cm,1.5cm)	
	\Huge \textcolor{black}{Weiran Zhang} 	\small Alter xx,  xxxx. %\textcolor{color1}{\textbf{Entwicklungsingenieur}}
	
	%	\Large \textbf{Entwicklung, Simulation}\\[0.8em]
	%	\large{\textbf{Key Skills}}: Stochastic Modelling, Computational Mechanics \normalsize
	%\\	
 \vspace{0em}
 Fachliche Schwerpunkte: Softwareentwicklung \& Numerische Simulation. \\[0.5em]
  	\href{https://www.xing.com/profile/Weiran_Zhang4/cv}{\faXing~\underline{Xing}}\hspace{1em}\href{https://www.linkedin.com/in/weiran-zh/}{\faLinkedinSquare~\underline{Linkedin}}\hspace{1em}\href{weiran.de}{\faWordpress~\underline{weiran.de}}\hspace{0.7em}\href{https://github.com/joywezen/}{\faCodeFork ~\underline{GitHub}}\\[0.1em] \faMapMarker~ xxxxx,D-xxxx,xxxx\\
	\hspace{0em}\href{mailto:weiran-zhang@outlook.com}{\faEdit~weiran-zhang@outlook.com}\hspace{1em}\faPhoneSquare~+49~(0)176-2136 5452\\


\end{textblock*}
\begin{textblock*}{21cm}(2cm,4.8cm)	
	%\textbf{Soft-Skills:}~ $\bullet$ Service- und Teamorientiert ~ $\bullet$ Lernbereit\\
	%	\textbf{Fachkompetenzen:}~ $\bullet$ Stochastik ~ $\bullet$ Finite-Elemente
\end{textblock*}


\begin{textblock*}{21cm}(14.7cm,1.5cm)	
	%\includegraphics[scale=0.055]{picture2}
	%\noindent\shadowimage[scale=0.055]{picture2}\par\bigskip
	\begin{tikzpicture}[overlay]
	\node[opacity=1] at (3,-1.2)%%2.5,-2.3
	{\includegraphics[scale=0.9]{../../signature/DSC_3152/DSC_3152_480_Blue}};
	\end{tikzpicture}
\end{textblock*}
\begin{textblock*}{21cm}(16.3cm,25cm)	
	%\includegraphics[scale=0.055]{picture2}
	%\noindent\shadowimage[scale=0.055]{picture2}\par\bigskip
	\begin{tikzpicture}
	%\node[opacity=0.1] {\includegraphics[scale=1]{Bosch.eps}};
	\end{tikzpicture}
\end{textblock*}
\vspace*{5em} 	
\begin{minipage}{1\linewidth}
	\vspace{1em}
	\textcolor{color1}{\Large \textbf{Arbeitserfahrungen}}
	\vspace{1em}
\end{minipage}
	%\section{Arbeitserfahrungen}
		\cventrynocomma{}{Projektingenieur -  Zahnmedizintechnik}{ %\href{https://www.d-touch.eu/}{\small \faAngleDoubleRight \underline{D-Touch} }
			 \hfill \small \underline{seit 02/2020}~~~~~}{TopDent GmbH, Hamburg, Deutschland}{}{
		\begin{itemize}
			%	\item [\textcolor{color1}{$\bullet$}] {\textbf{Data-Driven Marketing - Zahnmedizintechnik}}
			%	Datenbereinigung, Mustererkennung, Datenvisualisierung
			%\item \textcolor{color3}{{Anforderungsmanagement, Produktionsplanung} }
			\item \textcolor{color1}{\textbf{Anforderungsmanagement}} und Produktinnovation nach EN14683-II-R. 
			\item   \textcolor{color1}{\textbf{Koordination}} zwischen \textcolor{color1}{\textbf{EU}} und \textcolor{color1}{\textbf{Asien}}, einschl. Rohstoffeinkauf, Qualitätssicherung (TÜV Rheinland). 
			%\item Website aufbauen
			%	\item Unterstützte Geschäftsbereiche: Zahnmedizintechnik, Werkzeugstahl. 
	\end{itemize}}
%	\cventrynocomma{}{Technischer Berater -  Hochleistungsstahl}{
%		%~\href{https://www.ts-europe.de/}{\small \faAngleDoubleRight\underline{TS-Steels}}
%		\hfill \small \underline{seit 11/2019}~~~~~}{\textit{- Freiberuflich}, Taisheng Europe GmbH, Neuenkirchen, Deutschland}{}{
%		\begin{itemize}
%			%	\item [\textcolor{color1}{$\bullet$}] {\textbf{Data-Driven Marketing - Zahnmedizintechnik}}
%			%	Datenbereinigung, Mustererkennung, Datenvisualisierung
%			\item \textcolor{color1}{\textbf{Kundenberatung}} mitteleuropäischer   Ölindustrie.
%			%	\item[\textcolor{color1}{$\bullet$}]\textcolor{color1}{\textbf{Internetmarketing, Kundenberatung Mitteleuropa} }
%			%	\item Internetmarketing, Kundenberatung mitteleuropäischer  Ölindustrie. 
%			\item Unterstützung für  Vertriebsteam bei Messe Düsseldorf. 
%			%	\item Unterstützte Geschäftsbereiche: Zahnmedizintechnik, Werkzeugstahl. 
%	\end{itemize}}
	\cventrynocomma{}{Wissenschaftlicher Mitarbeiter}{
		%~\href{https://www.ibnm.uni-hannover.de/de/institut/news-und-veranstaltungen/detail/news/verteidigung-der-doktorarbeit-3/}{\small \faAngleDoubleRight\underline{Link}}
		\hfill\small \underline{11/2016 - 10/2019}}{Institut für Baumechanik und Numerische Mechanik, Uni Hannover, Deutschland}{}{
	%\textcolor{color1}{$\bullet$ \hspace{0.33em}\textbf{Funktionsentwicklung CAE-Software}}
	%, \textbf{Numerische mechanics}	
	\begin{itemize}%
		%	\item Funktionsentwicklung \textcolor{color1}{\textbf{CAE-Software, Lehre, Koordination} }
		%\item[\textit{Tasks}:] %\begin{itemize} DFG founded international training project	
		%\item Zuverlässigkeitsschätzung von Werkstoffstrukturen mittels Monte-Carlo-Simulation.  %stochastische Prozess  
		%\item Funktionsentwicklung \textcolor{color1}{\textbf{Berechnungssoftware}} für \textcolor{color1}{\textbf{Sicherheitsvorhersage}}.
		\item \bluehighlight{Funktionsentwicklung} für Sicherheits- und Lebensdauervorhersage von Bauteilen. 
				\item Wöchentlicher  \bluehighlight{Review} in CAE-Softwareentwicklung für Fatigue-Berechnungsmodul.
		%	\item Dokumentation von Forschungsprogress und Erstellung technischer Berichten. 
		%	\item Modellierung thermodynamischer Prozessen durch Zufallsprozess.
		%
		%	(z.B. 3Mon. gedauert Experiment wird innerhalb 10Min. simuliert.)  
		%\item Monte-Carlo simulation of stochastic process for structure reliability assessment.
		%\item High performance  non-linear FE analysis for high-cycle fatigue (one million cycles in minutes).
		%\item Large scale data visualisation.
		%			\item  with life-cycle prediction.
		%		\end{itemize} $\bullet$\hspace{0.63em}
		%	\item [\textcolor{color1}{ $\bullet$}]\textcolor{color1}{\textbf{Betreuung von Studierenden, Koordination} }
		\item Tutor der Vorlesung "\textcolor{color1}{\textbf{Stochastische Finite-Elemente-Methode}}" (SFEM).
		%	\item Als Koordinator zwischen den Forschungsgruppen aus Hannover und Paris.			
		%		\item Betreute zwei Bachelorarbeiten und eine Masterarbeit.
		\item Als \textcolor{color1}{\textbf{Koordinator}} zwischen den Forschungsgruppen in Hannover und Paris.	
		%	\item Als Koordinator für Seminar 2017 (Wernigerode), Winter-School 2018 (Karlsruhe).  
	\end{itemize}
}

%\cventrynocomma{}{Wissenschaftlicher Mitarbeiter}{\hfill \normalfont \underline{11/2016 - 10/2019}}{Institut für Baumechanik und Numerische Mechanik, Uni Hannover, Deutschland}{}{
%	\textbf{$\bullet$\hspace{0.63em}Funktionsentwicklung (V-Modell) für CAE-Software}
%	%, \textbf{Numerische mechanics}	
%	\begin{itemize}%
%		%\item[\textit{Tasks}:] %\begin{itemize} DFG founded international training project	
%		%\item Zuverlässigkeitsschätzung von Werkstoffstrukturen mittels Monte-Carlo-Simulation.  %stochastische Prozess  
%		\item Modellentwicklung für Sicherheits- und Lebensdauervorhersage von Bauteilen. 
%		\item Integration neuer Berechnungsmodule in bestehender Finite-Elemente-Software.  
%		\item Dokumentation von Forschungsprogress und Erstellung technischer Berichten. 
%	%	\item Modellierung thermodynamischer Prozessen durch Zufallsprozess.
%		%
%		%	(z.B. 3Mon. gedauert Experiment wird innerhalb 10Min. simuliert.)  
%			%\item Monte-Carlo simulation of stochastic process for structure reliability assessment.
%			%\item High performance  non-linear FE analysis for high-cycle fatigue (one million cycles in minutes).
%			%\item Large scale data visualisation.
%%			\item  with life-cycle prediction.
%%		\end{itemize} 
% 	\end{itemize}
% 	\textbf{$\bullet$\hspace{0.63em}Koordination, Betreuung von Studierenden}
% 	\begin{itemize}
%% 		\item  für die Forschungsgruppe 
% 		\item Als Koordinator zwischen den Forschungsgruppen aus Hannover und Paris.
% 		\item zzgl. für die Veranstaltungen Seminar 2017 (Wernigerode), Winter-School 2018 (Karlsruhe). 
% 		\item Betreute zwei Bachelorarbeiten und eine Masterarbeit.
% 	\end{itemize}
%}
 
 \cventrynocomma{}{Gastforscher}{
 	%~\href{http://lmt.ens-paris-saclay.fr}{\small \faAngleDoubleRight\underline{Link}}
 	\hfill\small \underline{01/2019 - 07/2019}}{Laboratoire de Mécanique et Technologie, Universit\'e Paris-Saclay, Frankreich}{}{
	% \textcolor{color1}{$\bullet$ \hspace{0.33em}\textbf{} }
	\begin{itemize}
		%	\item [\textcolor{color1}{ $\bullet$}]\textcolor{color1}{\textbf{Funktionstest, FEM} }
		\item \textcolor{color1}{\textbf{Algorithmen- und Integrationstest}} durch Quasi-Monte-Carlo-Verfahren auf HPC.
		%	\item Finite-Elemente-Berechnung der Ermüdungsschädigung. 
		%	\item  Analyse der Rissentwicklung unter zyklischer Belastung durch Ultraschalluntersuchung.
		%\item Parallelisierte Berechnung von lokalen FE-Moduls, Extrapolation von Zeitreihendaten. % Distributed computing, parallelisation of local FE module,  model extrapolation. 
		\item \textcolor{color1}{\textbf{Modellkalibrierung}} anhand \textcolor{color1}{\textbf{statistischer Versuchsplanung}} mit Ultraschallmessverfahren. 		
\end{itemize}}
\cventrynocomma{}{Wissenschaftliche Hilfskraft}{%~\href{https://www.tu-braunschweig.de/wire/forschung}{\small \faAngleDoubleRight\underline{Link}}
	\hfill\small\underline{03/2015 - 03/2016}}{Institut für  Wissenschaftliches Rechnen, TU Braunschweig, Deutschland}{}{
	\begin{itemize}%  
		%\item Sensordatenverarbeitung mittels 4D-Variation und Ensemble-Kalman-Filter (EnKF).
		%\item [\textcolor{color1}{ $\bullet$}] \textcolor{color1}{\textbf{Sensor-Fusion, Embedded-System} }		
		\item \textcolor{color1}{\textbf{Sensordatensimulation}} mit Opensource-Software (FEAP, UC Berkeley).
		\item \bluehighlight{Optimierung} der Modellparametern durch Anwendung von Bayes-Statistik.	
		\item  \textcolor{color1}{\textbf{Signalverarbeitung}} mittels 4D-Variation und Ensemble-Kalman-Filter (EnKF).
		%	\item Sicherheitsüberwachung von Bauteilen mit ARM-Board und Dehnungssensor.
		%	\item Arbitrary-Lagrangian-Eulerian-Verfahren (ALE) für gekoppelte Feldprobleme.
		%Referenzlösung erstellt mittels Open-Source-FEA-Software FEAP (entwickelt von UC Berkeley).%Geometrische Parameter-Identifikation aus einer Stahlplatte unter Spannung und Kompressionstest. Das Ziel war um die Position und Größe einer kreisförmigen Aufnahme in einer Stahlplatte zu identifizieren. Der Stand und Messdaten wurden auf der Open-Source Finite-Elemente Software FEAP (UC Berkeley) simuliert, doch der Posterior wurde durch EnKF identifiziert.  %For the simulation is used the ensemble Kalman filter and FE software FEAP (UC Berkeley).
\end{itemize}}

%\cventrynocomma{}{Prozessingenieur~}{ \small\textit{- Praktikum},  Yutong Bus, Zhengzhou, China %~\href{https://en.yutong.com/}{\small \faAngleDoubleRight\underline{Link}} \normalsize 
%	\hfill  \underline{06/2013 - 09/2013}}{}{}{\begin{itemize}%
%		\item Auftragsplanung \textcolor{color1}{\textbf{statistischer Qualitätskontrolle}}  des Lackierungsprozesses.
%		%		\item Oberflächeninspektion und Protokollieren des Lackierungsprozesses.
%\end{itemize}}
%\cventrynocomma{}{Computertechniker~}{\small -\textit{Teilzeit}, Hasee Computer, Zhengzhou, China \hfill  \underline{01/2011 - 06/2013}}{}{}{
%	\begin{itemize}%
%		\item Os- und Hardwarewartung des \textcolor{color1}{\textbf{Rechnerzentrums}}.
%%\item Aufbauen von Sicherheitssystemen mit Kamera und Bewegungsmelder.
%	\end{itemize}
%}
%\cventry{\normalfont \underline{07/2009 - 06/2013}}{IT-Administrator}{Teilzeit}{Yuke Technology}{Zhengzhou, China}{
%	\begin{itemize}%
%		\item Systemwartung Linux-Server.
%\end{itemize}}
\begin{minipage}{1\linewidth}
	\vspace{1em}
	\textcolor{color1}{\Large \textbf{Ausbildungen}}
	\vspace{1em}
\end{minipage}
%\section{Ausbildungen}
\cventrynocomma{}{\textbf{Promotion in Numerischer Mechanik}}{\hspace{0.5em}\small\textit{Note: sehr gut} \hfill  \underline{11/2016 - 11/2020}}{Leibniz Universität Hannover, Deutschland}{}{
\begin{itemize}%	
	\item[$\bullet$] \textit{Dissertation: "Stochastishe Modellierung und Numerische Simulation von Ermüdungsschädigung".}
	\item  Paralleles Rechnen, Stochastischer Prozess, Finite-Elemente-Analyse (FEA).
%	\item , .
	%	\item Im Vergleich mit klassischen Monte-Carlo, bezüglich die Berechnungskosten.
\end{itemize}
}
\cventrynocomma{}{M.Sc. \textbf{Computational Sciences in Engineering (CSE)}}{\hspace{0.5em}\small\textit{Note: 2,1} \hfill \underline{10/2013 - 05/2016}}{}{\newline Technische Universität Braunschweig, Deutschland}{
	\begin{itemize}
		\item Numerik der Differentialgleichungen, Linear- und nichtlinearen Systemen. 
		\item Bayessche Optimierung, Regelungstechnik, Werkstoffmechanik.
		%	\item Masterarbeit \textit{Note:\textbf{1,3}}, Projektarbeit \textit{Note:\textbf{1,0}}.
		%	\hspace{-1em}*\hspace{1em}  \item gute Kenntnisse in Datenanalyse und Verarbeitung.
\end{itemize}} 
%\begin{textblock*}{2cm}(2cm,22.5cm)
%	*  
%\end{textblock*}
%\footnotetext{* Arbeitszeugnisse wurden beigefügt.} 
%\footnotetext{* Dissertation wurde am 16.06.2020 abgegeben.}
%\hspace*{-0.4cm}
% arguments 3 to 6 can be left empty
\cventrynocomma{}{B.Ing. \textbf{Qualität- und Zuverlässigkeit-Ingenieurwissen}}{\hspace{0.5em}\small\textit{GPA: 83/100} \hfill  \underline{09/2009 - 06/2013}}{}{\newline Zhengzhou Institute of Aeronautical Industrial Management, China}{\begin{itemize}
		\item Toyota-Produktion-System (TPS), Six-Sigma (6$\sigma$), Fehleranalyse.
		%	\item  Total-Quality-Management (TQM).
	%	\item Datenvisualisierung, Fehlerbaum.  
		%		\item , ERP.
\end{itemize}}
%\newpage
%\vspace{-2em}
\begin{minipage}{1\linewidth}
	\vspace{1em}
	\textcolor{color1}{\Large \textbf{Fähigkeiten und Interessen}}
	\vspace{1em}
\end{minipage}
%	\section{Weitere Fähigkeiten und Interessen}
\cvitem{Software}{Matlab, Git, Docker, DOORS, Jira, CAD, Abaqus, Ansys (mit Python-script).}{}{}
\cvitem{Coding}{Python (Pandas, Scipy, TensorFlow, GPIO), C++ (PETSc, QT, CUDA).}
%							\cvitem{\hspace{-1cm}Programming}{Matlab, Python}{}{}
\cvitem{Sprachen}{Deutsch (fließend), English (fließend), Mandarin (Muttersprache).}{}{}
%	\cvitem{Sprachen}{English (Arbeitssprache), Chinese (Muttersprache), French \& Portugiese (Grundkenntnisse) }{}{}
%		\cvitem{}{\hspace{5.2em}EU/China Führerschein vorhanden (ohne Verkehrsverstöße seit 5 Jahren)}{}{}
%\cvitem{Licenses}{EU and chinese driving license}{}{}
\cvitem{Hobbys}{Saxophon (Sopran \& Tenor), Mountain Biking, Reparaturarbeiten.}{}{}
\clearpage
\begin{minipage}{1\linewidth}
%	\vspace{1em}
	\textcolor{color1}{\Large \textbf{Studien- und Abschlussprojekten}}
	\vspace{1em}
\end{minipage}
%\section{Studien- und Abschlussarbeiten}
%\hspace*{-0.4cm}
\cventry{\small\normalfont \underline{10/2015 - 04/2016}}{Masterarbeit}{Note: {1,3}}{}{TU Braunschweig}{
	\textit{\textbf{$\bullet$}}\hspace{0.63em}\textbf{Quantifizierung von Unsicherheit} des Schadens im Betonträger durch\\ Multi-Level-Monte-Carlo-Verfahren
	\begin{itemize}%
		\item Modellierung der Materialeigenschaften mittels \textcolor{color1}{\textbf{Zufallsfelder}}.
			\item \textcolor{color1}{\textbf{Sicherheitsprognosen}} von Stahlbeton anhand Finite-Elemente-Berechnung.
	%	\item Im Vergleich mit klassischen Monte-Carlo, bezüglich die Berechnungskosten.
\end{itemize}} 
%\footnotetext{* Demonstriert im beigefügten Portfolio (in englischer Sprache).}
\cventry{\small\normalfont \underline{11/2014 - 05/2015}}{Projektarbeit} {Note: {1,0}}{}{TU Braunschweig}{
	\textit{\textbf{$\bullet$}}\hspace{0.63em}\textbf{Sensor-Fusion} durch Ensemble-Kalman-Filter und 4D-Variationsmethode
	\begin{itemize}%
		\item Vergleicht  beider Methoden von der Effizienz und Genauigkeit in \textcolor{color1}{\textbf{Datenassimilation}}.
		\item  Lösen mehrdimensionales chaotisches System durch 4-stufigen \textcolor{color1}{\textbf{Runge-Kutta}} Verfahren.
\end{itemize}}
\cventry{\small\normalfont \underline{12/2012 - 05/2013}}{Bachelorarbeit}{Note: {95/100}}{\newline Zhengzhou Institute of Aeronautical Industrial Management}{}{\textit{\textbf{$\bullet$}}\hspace{0.63em}\textbf{Qualitätsoptimierung} von Bankdienstleistungen aus technischer Perspekt.
	\begin{itemize}%
		\item \textcolor{color1}{\textbf{Hauptkomponentenanalyse}} des multivariaten Bewertungsmodells der Servicequalität.
		\item Topologiebasierte Servicenetzwerk-Optimierung, Erhöhung der \bluehighlight{HMI-Barrierefreiheit}.
	%	\item  Ergonomische Verbesserung der Zugänglichkeit von Geldautomaten.
\end{itemize}}

%\begin{minipage}{1\linewidth}
%		\vspace{1em}
%	\textcolor{color1}{\Large \textbf{Weitere Fähigkeiten und Interessen}}
%	\vspace{1em}
%\end{minipage}
%%	\section{Weitere Fähigkeiten und Interessen}
%\cvitem{Software}{Git, Docker, Matlab, CAD, Abaqus, Ansys (mit Python-script).}{}{}
%\cvitem{Coding}{Python (Pandas, Scipy, TensorFlow, GPIO), C++ (PETSc, QT, CUDA).}
%%							\cvitem{\hspace{-1cm}Programming}{Matlab, Python}{}{}
%\cvitem{Sprachen}{Deutsch (fließend), English (fließend), Mandarin (Muttersprache).}{}{}
%%	\cvitem{Sprachen}{English (Arbeitssprache), Chinese (Muttersprache), French \& Portugiese (Grundkenntnisse) }{}{}
%%		\cvitem{}{\hspace{5.2em}EU/China Führerschein vorhanden (ohne Verkehrsverstöße seit 5 Jahren)}{}{}
%%\cvitem{Licenses}{EU and chinese driving license}{}{}
%\cvitem{Hobbys}{Saxophon (Sopran \& Tenor), Mountain Biking, Fotografieren.}{}{}
%Weitere Fähigkeiten und Interessen
\begin{minipage}{1\linewidth}
	\vspace{1em}
	\textcolor{color1}{\Large \textbf{Vorträge}}
	\vspace{1em}
\end{minipage}
%\section{Vorträge bei internationalen akademischen Veranstaltungen}
%\cventry{\normalfont (Dez. 2017)}{Senlis, Frankreich}{}{7th International Conference Fatigue Design}{}{}
\cventry{\small\normalfont \underline{05/2018}}{Poitier, Frankreich}{}{12th International Fatigue Congress}{}{\textit{Vortrag: On the time discritisation effect of stochastic damage evolution.}}
\cventry{\small\normalfont \underline{06/2019}}{Crete, Griechenland}{}{3rd International Conference on Uncertainty Quantification \newline in Computational Sciences and Engineering}{}{\textit{Vortrag: Stochastic modelling of fatigue process.}}
\cventry{\small\normalfont \underline{09/2019}}{Barcelona, Spanien}{}{15th International Conference on Computational Plasticity}{}{\textit{Vortrag: Modelling approach and efficient numerical scheme for stochastic fatigue process.}}
\begin{minipage}{1\linewidth}
	\vspace{1em}
	\textcolor{color1}{\Large \textbf{Veröffentlichung}}
	\vspace{1em}
\end{minipage}
%\section{Veröffentlichung}
\small Stochastic Material Modeling for Fatigue Damage Analysis. \textit{W.Zhang, A.Fau, U.Nackenhorst, R.Desmorat}
 In: Virtual Design and Validation. Springer, Cham, 2020. S. 329-347. \\
% \begin{minipage}{1\linewidth}
% 	\vspace{1.5em}
% 	\textcolor{color1}{\Large \textbf{Auszeichnungen}}
% 	\vspace{1em}
% \end{minipage}
%%\section{Auszeichnungen}
%\cventry{\small\normalfont \underline{11/2013}}{CSE "Junior"-Stipendium}{}{}{}{Technische Universität Braunschweig. Braunschweig, Deutschland}
%\cventry{\small\normalfont \underline{06/2013}}{"Ausgezeichneter Absolvent"}{}{}{}{Zhengzhou Institute of Aeronautical Industrial Management. Zhengzhou, China}
%\cventry{\small\normalfont \underline{10/2012}}{Stipendium Jahr 2012}{}{}{}{Zhengzhou Institute of Aeronautical Industrial Management. Zhengzhou, China}
%\hspace{-2em}*\hspace{2em} 
%, EU and Chinese driving license
%\begin{textblock*}{2cm}(2cm,18.1cm)
%*  
%\end{textblock*}
% arguments 3 to 6 can be left empty

%\cventry{\normalfont \underline{09/2011}}{1. Platz bei studentischem Karriereplanwettbewerb}{}{}{}{Zhengzhou Insitute of Aeronautical Industrial Management. Zhengzhou, China}
%\cventry{\normalfont \underline{06/2011}}{2. Platz bei studentischem Musikwettbewerb der Provinz Henan}{}{}{}{Musician Association of Henan Province, Zhengzhou, China}
%\section{Vorträge bei internationalen akademischen Veranstaltungen}
%%\cventry{\normalfont (Dez. 2017)}{Senlis, Frankreich}{}{7th International Conference Fatigue Design}{}{}
%\cventry{\normalfont \underline{05/2018}}{Poitier, Frankreich}{}{12th International Fatigue Congress}{}{\textit{Vortrag: On the time discritisation effect of stochastic damage evolution.}}
%\cventry{\normalfont \underline{06/2019}}{Crete, Griechenland}{}{3rd International Conference on Uncertainty Quantification \newline in Computational Sciences and Engineering}{}{\textit{Vortrag: Stochastic modelling of fatigue process.}}
%\cventry{\normalfont \underline{09/2019}}{Barcelona, Spanien}{}{15th International Conference on Computational Plasticity}{}{\textit{Vortrag: Modelling approach and efficient numerical scheme for stochastic fatigue process.}}
%\section{Veröffentlichung}
%Stochastic Material Modeling for Fatigue Damage Analysis. \textit{W.Zhang, A.Fau, U.Nackenhorst, R.Desmorat}\newline In: Virtual Design and Validation. Springer, Cham, 2020. S. 329-347. [In englischer Sprache] 
							
							\flushright
							\vspace{1em}
							Braunschweig, \today\\[1em]	 \includegraphics[width=0.15\textwidth]{../../signature/zhang_signature.png}
							\footnote{Dieser Lebenslauf wurde mit \LaTeX ~in Buchdruckqualität erstellt. Quellcode erhältlich in meinem \href{https://github.com/joywezen/}{\underline{GitHub}}.
							}	
								\clearpage
